%Copyright 2014 Jean-Philippe Eisenbarth
%This program is free software: you can 
%redistribute it and/or modify it under the terms of the GNU General Public 
%License as published by the Free Software Foundation, either version 3 of the 
%License, or (at your option) any later version.
%This program is distributed in the hope that it will be useful,but WITHOUT ANY 
%WARRANTY; without even the implied warranty of MERCHANTABILITY or FITNESS FOR A 
%PARTICULAR PURPOSE. See the GNU General Public License for more details.
%You should have received a copy of the GNU General Public License along with 
%this program.  If not, see <http://www.gnu.org/licenses/>.

%Based on the code of Yiannis Lazarides
%http://tex.stackexchange.com/questions/42602/software-requirements-specification-with-latex
%http://tex.stackexchange.com/users/963/yiannis-lazarides
%Also based on the template of Karl E. Wiegers
%http://www.se.rit.edu/~emad/teaching/slides/srs_template_sep14.pdf
%http://karlwiegers.com


\documentclass{scrreprt}

\usepackage{pdfpages}
\usepackage{listings}
\usepackage[T1]{fontenc}
\usepackage{placeins}
\usepackage{float}
\usepackage{underscore}
\usepackage[bookmarks=true]{hyperref}
\usepackage[utf8]{inputenc}
\usepackage{graphicx}
\usepackage{subfigure}
\usepackage[english]{babel}
\hypersetup{
	bookmarks=false,    % show bookmarks bar?
	pdftitle={Software Requirement Specification},    % title
	pdfauthor={Adam-Ryan},                     % author
	pdfsubject={TeX and LaTeX},                        % subject of the document
	pdfkeywords={TeX, LaTeX, graphics, images}, % list of keywords
	colorlinks=true,       % false: boxed links; true: colored links
	linkcolor=blue,       % color of internal links
	citecolor=black,       % color of links to bibliography
	filecolor=black,        % color of file links
	urlcolor=blue,        % color of external links
	linktoc=page            % only page is linked
}%
\def\myversion{1 }
\date{}
%\title
\usepackage{hyperref}
\begin{document}
	
	\begin{flushright}
		\rule{16cm}{5pt}\vskip1cm
		\begin{bfseries}
			\Huge{Tutorial 2\\}
			\vspace{1.9cm}
			for\\
			\vspace{1.9cm}
			Ruby Exploration
			\vspace{1.9cm}
			\LARGE{Version \myversion}\\
			\vspace{1.9cm}
			Adam Ryan (14395076)\\
			\vspace{1.9cm}
			COMP47530\\
			\vspace{1.9cm}
			\today\\
		\end{bfseries}
	\end{flushright}
	
	\tableofcontents
	
	\chapter{Questions}\label{Intro}
	
	
	\section{Exercise 1 - Questions}\label{E1Q}
	In irb and each of the primitives class and instance\_of? test the following to see types of object they are and explain why you get the answers you do:
	\begin{enumerate}
		\item "hello there big boy"
		\item $56$
		\item 34.00 
		\item 0.222222354454365 
		\item ("a", "b", "c")
		\item +
		\item PI 
		\item Math::PI 
		\item add 
		\item hellow
		\item hello = 8 and then check hello with class "goodbye"
		\item (56 + 45.32)
		\item (56 + 45)
		\item 5.to\_s 
		\item "5".to\_i 
		\item five.to\_s
	\end{enumerate}
	
	\section{Exercise 1 - Answers}\label{E1A}
	Answers
	\begin{enumerate}
		\item "hello there big boy" is a string because it's a string.
		\item $56$ is an Integer as it has no trailing decimals and is not a string.
		\item 34.00  is a float
		\item 0.222222354454365 is a float as it has a decimal which is non-zero. 
		\item ("a", "b", "c") is an array
		\item + returns a SyntaxError with .class because it does not have a class variable accessible
		\item PI  returns uninitialized constant PI (NameError) because it hasn't been initialised.
		\item Math::PI returns a float and is 3.14...
		\item add returns an undefinied local variable when checking its class
		\item hellow returns an uninitialised variable as it hasn't yet been set
		\item hello = 8 and then check hello with class returns an integer because we've set the hello variable to be an integer with value 8.
		\item "goodbye" returns a string because it's a string.
		\item (56 + 45.32) returns a float because one of the values being added is a float
		\item (56 + 45) returns an integer because both values are integers
		\item 5.to\_s returns a string.
		\item "5".to\_i  returns an integer
		\item five.to\_s returns a variable not initialised error becuase five hasn't been defined.
	\end{enumerate}
	
	\section{Exercise 2 - Questions} \label{E2Q}
	The following details the questions in section 2.
	\begin{enumerate}
		\item "hello there big boy".include?("boy")
		\item "hello there big boy".include(" big")
		\item "hello there big boy".include?(" ere")
		\item What happens when you evaluate: ["a", "b", "c"] + ["d"]
		\item What happens when you evaluate: ["a", "b", "c"] + "d"
		\item Is there an easy way to capitalise words, so "hello" becomes "Hello" ?
		\item In the same vein, make "hello" "HELLO".
		\item Write a command to print out your name.
		\item Write a method to print out your name.
		\item Write a method to print out any name.
		\item Set up the varibles, maxi, dick and twinko so that they are all assigned numbers but two of them are assigned to the same numbers. Then show with a series of equality tests which ones actually have the same value.
		\item If you change the variables with the same number to be a Float and Fixnum does it change the results of the equality tests ?
		\item Do a version of these test using strings rather than numbers.
	\end{enumerate}
	
	\chapter{Exercise 2 -  Answers}\label{E2A}
	
	\begin{enumerate}
		\item "hello there big boy".include?("boy") - Returns true 
		\item "hello there big boy".include(" big") - Returns an error because it's the wrong function name.
		\item "hello there big boy".include?(" ere") - Returns false because of the space
		\item What happens when you evaluate: ["a", "b", "c"] + ["d"]  - It adds all elements in the second list into the first list; that is to say it unions the two.
		\item What happens when you evaluate: ["a", "b", "c"] + "d" - It returns a type error as + isn't defined for adding two different types (string and a list)
		\item Is there an easy way to capitalise words, so "hello" becomes "Hello" ? - "hello".capitalize
		\item In the same vein, make "hello" "HELLO". - "hello".upcase
		\item Write a command to print out your name. - puts "Adam Ryan"
		\item Write a method to print out your name. 
		\begin{verbatim}
			def my_name
				puts "Adam Ryan"
			end 
			
			my_name
		\end{verbatim}
		\item Write a method to print out any name.
		
		\begin{verbatim}
			def a_name(name)
				puts name
			end 
			
			a_name("Adam Ryan")
	\end{verbatim}

		\item Set up the variables, maxi, dick and twinko so that they are all assigned numbers but two of them are assigned to the same numbers. Then show with a series of equality tests which ones actually have the same value.
		
		\begin{verbatim}
			maxi=1
			dick=2
			twinko=1
			
			maxi===maxi Returns True
			maxi===dick Returns False
			maxi===twinko Returns True
			
						
			dick===maxi Returns False
			dick===dick Returns True
			dick===twinko Returns False
			
						
			twinko===maxi Returns True
			twinko===dick Returns False
			twinko===twinko Returns True
			
		\end{verbatim}
		\item If you change the variables with the same number to be a Float and Fixnum does it change the results of the equality tests ? - Fixnum is depreciated, however due to rounding there can be instances where the same number as a float and fixnum will not evaluate as equal.

		\item Do a version of these test using strings rather than numbers. -  When the test is done as strings, setting one variable to "5" and another to 5 results in these two not equalling. When both are set as strings then they equal.
	\end{enumerate}
	
		
	\section{Exercise 3 - Questions}
	What's a predicate?

	\section{Exercise 3 - Answers}
	Predicate methods in Ruby are those which end in a question mark to highlight that you are 'asking' a question to Ruby when calling these methods. These methods return true or false (and it is breaking convention to not do so).
		
	\section{Exercise 4 - Questions}
Define your own adding method that always adds 5 and 6 together.
So, my\_add\_five\_to\_six => 11.

	\section{Exercise 4 - Answers}
\begin{verbatim}
	def my_add_five_to_six
		5+6
	end 
	
	my_add_five_to_six
\end{verbatim}

		
\section{Exercise 5 - Questions}
Put this defined method in a file and call it using the ruby command outside of irb

		
\section{Exercise 5 - Answers}
Copy the above, paste it into a file called eleven.rb and call the file.
\end{document}
