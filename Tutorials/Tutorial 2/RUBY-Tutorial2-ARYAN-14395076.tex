%Copyright 2014 Jean-Philippe Eisenbarth
%This program is free software: you can 
%redistribute it and/or modify it under the terms of the GNU General Public 
%License as published by the Free Software Foundation, either version 3 of the 
%License, or (at your option) any later version.
%This program is distributed in the hope that it will be useful,but WITHOUT ANY 
%WARRANTY; without even the implied warranty of MERCHANTABILITY or FITNESS FOR A 
%PARTICULAR PURPOSE. See the GNU General Public License for more details.
%You should have received a copy of the GNU General Public License along with 
%this program.  If not, see <http://www.gnu.org/licenses/>.

%Based on the code of Yiannis Lazarides
%http://tex.stackexchange.com/questions/42602/software-requirements-specification-with-latex
%http://tex.stackexchange.com/users/963/yiannis-lazarides
%Also based on the template of Karl E. Wiegers
%http://www.se.rit.edu/~emad/teaching/slides/srs_template_sep14.pdf
%http://karlwiegers.com


\documentclass{scrreprt}

\usepackage{pdfpages}
\usepackage{listings}
\usepackage[T1]{fontenc}
\usepackage{placeins}
\usepackage{float}
\usepackage{underscore}
\usepackage[bookmarks=true]{hyperref}
\usepackage[utf8]{inputenc}
\usepackage{graphicx}
\usepackage{subfigure}
\usepackage[english]{babel}
\hypersetup{
	bookmarks=false,    % show bookmarks bar?
	pdftitle={TUTORIAL2Ruby},    % title
	pdfauthor={Adam-Ryan},                     % author
	pdfsubject={TeX and LaTeX},                        % subject of the document
	pdfkeywords={TeX, LaTeX, graphics, images}, % list of keywords
	colorlinks=true,       % false: boxed links; true: colored links
	linkcolor=blue,       % color of internal links
	citecolor=black,       % color of links to bibliography
	filecolor=black,        % color of file links
	urlcolor=blue,        % color of external links
	linktoc=page            % only page is linked
}%
\def\myversion{1 }
\date{}
%\title
\usepackage{hyperref}
\begin{document}
	
	\begin{flushright}
		\rule{16cm}{5pt}\vskip1cm
		\begin{bfseries}
			\Huge{Tutorial 2\\}
			\vspace{1.9cm}
			for\\
			\vspace{1.9cm}
			Ruby Exploration
			\vspace{1.9cm}
			\LARGE{Version \myversion}\\
			\vspace{1.9cm}
			Adam Ryan (14395076)\\
			\vspace{1.9cm}
			COMP47530\\
			\vspace{1.9cm}
			\today\\
		\end{bfseries}
	\end{flushright}
	
	\tableofcontents
	
	\chapter{Questions}\label{Intro}
	
	
	\section{Exercise 1 - Questions}\label{E1Q}
	In irb and each of the primitives class and instance\_of? test the following to see types of object they are and explain why you get the answers you do:
	\begin{enumerate}
		\item "hello there big boy"
		\item $56$
		\item 34.00 
		\item 0.222222354454365 
		\item ("a", "b", "c")
		\item +
		\item PI 
		\item Math::PI 
		\item add 
		\item hellow
		\item hello = 8 and then check hello with class "goodbye"
		\item (56 + 45.32)
		\item (56 + 45)
		\item 5.to\_s 
		\item "5".to\_i 
		\item five.to\_s
	\end{enumerate}
	
	\section{Exercise 2 - Questions} \label{E2Q}
	The following details the questions in section 2.
	\begin{enumerate}
		\item "hello there big boy".include?("boy")
		\item "hello there big boy".include(" big")
		\item "hello there big boy".include?(" ere")
		\item What happens when you evaluate: ["a", "b", "c"] + ["d"]
		\item What happens when you evaluate: ["a", "b", "c"] + "d"
		\item Is there an easy way to capitalise words, so "hello" becomes "Hello" ?
		\item In the same vein, make "hello" "HELLO".
		\item Write a command to print out your name.
		\item Write a method to print out your name.
		\item Write a method to print out any name.
		\item Set up the varibles, maxi, dick and twinko so that they are all assigned numbers but two of them are assigned to the same numbers. Then show with a series of equality tests which ones actually have the same value.
		\item If you change the variables with the same number to be a Float and Fixnum does it change the results of the equality tests ?
		\item Do a version of these test using strings rather than numbers.
	\end{enumerate}
	
	
		
	\section{Exercise 3 - Questions}
	What's a predicate?

	\section{Exercise 4 - Questions}
Define your own adding method that always adds 5 and 6 together.
So, my\_add\_five\_to\_six => 11.

		
\section{Exercise 5 - Questions}
Put this defined method in a file and call it using the ruby command outside of irb

		
\end{document}
